\documentclass[12pt]{article}
\usepackage[spanish]{babel}
\usepackage{natbib}
\usepackage{url}
\usepackage[utf8x]{inputenc}
\usepackage{amsmath}
\usepackage{float}
\usepackage{subfig}
\usepackage{graphicx}
\graphicspath{{images/}}
\usepackage{parskip}
\usepackage{fancyhdr}
\usepackage{vmargin}
\setmarginsrb{3 cm}{2.5 cm}{3 cm}{2.5 cm}{1 cm}{1.5 cm}{1 cm}{1.5 cm}

\title{\#2 Programas}							% Título
\author{Alejandro Guirado García}											% Autores
\date{\today} %Aquí pueden cambiarla%					% Fecha de edición

\makeatletter
\let\thetitle\@title
\let\theauthor\@author
\let\thedate\@date										
\makeatother

\pagestyle{fancy}
\fancyhf{} %Si quieres ponerle otro encabezado/pie de pagina%
\lhead{\thetitle}
\cfoot{\thepage}

\begin{document}

%%%%%%%%%%%%%%%%%%%%%%%%%%%%%%%%%%%%%%%%%%%%%%%%%%%%%%%%%%%%%%%%%%%%%%%%%%%%%%%%%%%%%%%%%

\begin{titlepage}
	\centering
    \vspace*{0.5 cm}
    \includegraphics[scale = 0.5]{logo}\\[0.5 cm]	% University Logo
    \textsc{\Large Universidad de Sonora}\\[1.0 cm]	% University Name
	\textsc{\Large División de Ciencias Exactas y Naturales}\\[0.5 cm]				% Course Code
	\textsc{\large Física Computacional I}\\[0.5 cm]				% Course Name
	\rule{\linewidth}{0.2 mm} \\[0.4 cm]
	{ \huge \bfseries \thetitle}\\
	\rule{\linewidth}{0.2 mm} \\[0.5 cm]
 
	
	\begin{minipage}{\textwidth}
		\begin{flushleft} 
			\emph{\Large Integrante:} \large \\
			\theauthor
			\end{flushleft}
	\end{minipage}\\[1 cm]
	{\large \thedate}\\[2 cm]
 
	\vfill
	
\end{titlepage}

%%%%%%%%%%%%%%%%%%%%%%%%%%%%%%%%%%%%%%%%%%%%%%%%%%%%%%%%%%%%%%%%%%%%%%%%%%%%%%%%%%%%%%%%%

\section{Programas}
En esta actividad modificaremos algunos códigos para obtener otras incognitas. Se mostrarán los códigos inciales y los modificados. 
\subsection{Código \# 1}

En el código inicial se puede obtener la altura de un objeto que está cayendo de una torre de altura "h" (sin tomar en cuenta fuerzas de fricción), en un determinado tiempo.

\begin{verbatim}
h = float(input("Proporciona la altura de la torre: "))
t = float(input("Ingresa el tiempo: "))
s = 0.5*9.81*t**2
print("La altura de la pelota es", h-s, "metros")
\end{verbatim}

En el cógido modificado, se puede obtener el tiempo en que tardaría en llegar al suelo, soltado desde una altura "h",sin tomar en cuenta fuerzas de fricción.

\begin{verbatim}
h = float(input("Proporciona la altura de la torre: "))
t=sqrt(2*h/9.81)
print("El tiempo que tarda la pelota en llegar al suelo es", t ,"segundos")
\end{verbatim}


\subsection{Código \# 2}

Un satélite orbita la Tierra a una altura h, con un periodo T en segundos.

Demuestre que la altitud h del satélite sobre la superficie de la Tierra esta dado por la expresión

 
\begin{equation}
(R + h)^3 = (GMT^2)/(4 pi^2)
\end{equation}


 

donde G = 6.67 x 10-11 $m^3$ kg$^{-1}$ s$^{-2}$ es la constante de Gravitación Universal de Newton, M = 5.97 $\times 10^{24}$ kg es la masa de la Tierra y R=6371 km es su radio.
ESe pide un programa que pida al usuario ingresar el valor deseado de T y regrese la altura h correspondiente en metros.

\subparagraph*{Demostración}
\newcommand{\QED}{\hfill \textit{\textbf{Q.E.D.}}}
Si suponemos que los satélites recorren orbitas circulares.

\begin{equation}
\overrightarrow{F}=m\dot{a}=m\frac{\overrightarrow{v^2}}{\overrightarrow{r}}    
\end{equation}




fuerza responsable del movimiento orbital sería la fuerza gravitatoria por parte de la tierra , por lo tanto tomando en cuenta solamente los módulos, nos queda:

 \begin{equation}
\frac{GmM}{r^2}=m\frac{v^2}{r}
\end{equation}

Donde m es la masa de la tierra y M la masa del satélite.Se sabe que "r" es la distancia entre entre el centro de la tierra y el satélite.

Considerando el periodo del satélite:
\begin{equation}
 v=\frac{2\pi r}{T}
\end{equation}



Sustituyendo \label{4} en \label{3} y simplificando nos queda la siguiente expresión:

\begin{equation}
\frac{GM}{r^2}=\frac{4r\pi^2}{T}
\end{equation}

Por último, despejando a r, como r es la distancia del centro de la tierra al satélite se puede expresar r como (r'+h) donde r' sería el radio de la tierra y h la distancia de la superficie al satélite.
\begin{equation}
(R + h)^3 = (GMT^2)/(4 pi^2)
\end{equation}
Lo cual es lo que se quería demostrar.
\subparagraph{Código}
\begin{verbatim}
from math import pi

T = float(input("Proporciona el período de órbita en segundos: "))
k = (6.67e-11*5.97e24)/(4*pi*pi)
h = (k*T*T)**(1./3.)-6371000
print("La altura del objeto con respecto a la superficie terrestre es",
h, "metros")

\end{verbatim}
 
Después se nos pide, verificar la distancia del satélite a la tierra en base a su periodo, se mostrará en las siguientes 3 imagenes:

\begin{figure}[H]
   \centering
    \includegraphics[height=5cm]{84600.png}
    \caption{Con un periodo de 84600 segundos}
\end{figure}


\begin{figure}[H]
   \centering
    \includegraphics[height=5cm]{84600.png}
    \caption{Con un periodo de 5400 segundos}
\end{figure}


\begin{figure}[H]
   \centering
    \includegraphics[height=5cm]{3600.png}
    \caption{Con un periodo de 3600 segundos}
\end{figure}
\subsection{Código \# 3}

Coordenadas polares. Un punto en el espacio en el sistema de coordenadas polares se describe por las cantidades (r, $\theta$).

 

La relación entre coordenadas polares y el sistema de coordenadas cartesianos, esta dada por las ecuaciones:  x = r $\cos \theta$, y = $r \sin \theta$.

 El siguiente programa para calcular las coordenadas cartesianas a partir de las coordenadas polares: 
 
 \begin{verbatim}
from math import sin,cos,pi
r = float(input("Introduce r: "))
d = float(input("Ingresa theta en grados: "))
theta = d*pi/180
x = r*cos(theta)
y = r*sin(theta)
print("x =",x," y =",y)
\end{verbatim}

 Se nos pide modificar el código para obtener coordenas esféricas a partír de rectangulares.
 
 \begin{verbatim}
from math import pi,atan,acos
x = float(input("Introduce x: "))
y = float(input("Introduce y: "))
z = float(input("Introduce z: "))
r = sqrt(x*x+y*y+z*z)
theta = atan(r/z)
phi = acos(z/(sqrt((x*x)+(y*y)+(z*z))))
print("r =",r," theta =",theta," phi =",phi)
\end{verbatim}

Al dar los siguientes valores Introduce: x=10, y=15, z=20
r= 26.92582403567252, theta= 0.9319311825594854, phi = 0.7335813236400831.

\newpage
\subsection{Código \# 4}

El siguiente código nos permite ingresar un número, el cual si es par nos arrojará "even" y si es impar nos arrojará "odd".
\begin{verbatim}
n = int(input("Enter an integer: "))
if n%2==0:
     print("even")
else:
    print("odd")
\end{verbatim}

El siguiente código nos permite ingresar dos números y al final nos dice que dos números pusimos.

\begin{verbatim}
print("Enter two integers, one even, one odd.")
m = int(input("Enter the first integer: "))
n = int(input("Enter the second integer: "))
while (m+n)%2==0:
    print("One must be even and the other odd.")
    m = int(input("Enter the first integer: "))
    n = int(input("Enter the second integer: "))
print("The numbers you chose are",m,"and",n)
\end{verbatim}
\newpage
\subsection{Código \# 5}
 Los Números de Fibonacci  es una sucesión de números enteros aparecen en toda la naturaleza.  

El siguiente programa calcula la secuencia de Fibonacci, introduce la condición de control while 

\begin{verbatim}
f1,f2 = 1,1
while f2<1000:
     print(f2)
      f1,f2 = f2,f1+f2
      -----------------------------------
       

1
2
3
5
8
13
21
34
55
89
144
233
377
610
987

\end{verbatim}
Se busca que basados en está ídea se escriba un programa para los números del catalán, que son dados por la fórmula de recurrencia:  
\begin{equation}
C_0=1, C_{(n+1)} = \frac{2(2n+1)}{(n+2)}C_n
\end{equation}
\newpage
Imprimiendo cualquier numero menor o igual a 1000000.

\begin{verbatim}
n,c = 2,6

while c<100000:

       print(c)
 
             n,c = n+1,(((4*n)+2)/(n+2))*c
   -------------------------------
6
12
24
72
216
648
1944
5832
17496
52488

\end{verbatim}

\end{document}